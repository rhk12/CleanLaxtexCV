
\documentclass[a4paper,10pt]{article}
\usepackage[a4paper, margin=1in]{geometry}
\usepackage{amsmath,amssymb}
\usepackage{iftex}
\usepackage{enumitem} 

\ifPDFTeX
  \usepackage[T1]{fontenc}
  \usepackage[utf8]{inputenc}
  \usepackage{textcomp}
  \usepackage{newtxtext,newtxmath} % Times New Roman-like font for pdfLaTeX
\else
  \usepackage{fontspec} % Allows font customization
  \setmainfont{Times New Roman} % Set Times New Roman as main font for XeLaTeX/LuaLaTeX
\fi
                   
\usepackage{lmodern}
\ifPDFTeX\else
  % xetex/luatex font selection
\fi
\IfFileExists{upquote.sty}{\usepackage{upquote}}{}
\IfFileExists{microtype.sty}{%
  \usepackage[]{microtype}
  \UseMicrotypeSet[protrusion]{basicmath} % disable protrusion for tt fonts
}{}
\makeatletter
\@ifundefined{KOMAClassName}{%
  \IfFileExists{parskip.sty}{%
    \setlength{\parskip}
  }{
    \setlength{\parindent}{0pt}
    \setlength{\parskip}{6pt}}
}{
\KOMAoptions{parskip=half}}
\makeatother
\usepackage{xcolor}
\usepackage{tabularx}
\usepackage{longtable}
\usepackage{booktabs}
\setlength{\emergencystretch}{3em}
\providecommand{\tightlist}{%
  \setlength{\itemsep}{0pt}\setlength{\parskip}{0pt}}
\setcounter{secnumdepth}{-\maxdimen}
\ifLuaTeX
  \usepackage{selnolig}
  \setmainfont{Times New Roman}
\fi
\usepackage{bookmark}
\IfFileExists{xurl.sty}{\usepackage{xurl}}{}
\urlstyle{same}
\hypersetup{
  hidelinks,
  pdfcreator={LaTeX via pandoc}}

\author{}
\date{}

    \usepackage{titlesec}
    \usepackage{color}
    \titleformat{\subsection}
    {\normalfont\large\bfseries\color{black}} % format
    {\thesubsection} % label
    {1em} % sep
    {} % before-code
    \titleformat{\subsubsection}
    {\normalfont\normalsize\bfseries\color{black}} % format
    {\thesubsubsection} % label
    {1em} % sep
    {} % before-code
    
    \usepackage{fancyhdr}
    \usepackage{xcolor}
    \definecolor{darkgray}{gray}{0.4} % Define a darker gray color
    \fancypagestyle{firstpage}{
        \fancyhf{} % Clear all headers and footers first
        \rhead{\textcolor{darkgray}{December 2024}}
        \renewcommand{\headrulewidth}{0pt} % No header rule
    }
    \thispagestyle{firstpage}
    \begin{document}

    \begin{center}
    \LARGE \textbf{\textsc{REUBEN H.KRAFT}} \\
    \rule{\linewidth}{2pt}
    \end{center}
    \normalsize % Return to the default font size
    


    \section*{PROFESSIONAL POSITIONS}
    
    \subsection*{Academic}
    
                \noindent Professor of Mechanical and Biomedical Engineering, The Pennsylvania State University, University Park, PA, \textbf{2024 - Present}\vspace{0.25cm}
                
                \noindent Associate Professor of Mechanical and Biomedical Engineering, The Pennsylvania State University, University Park, PA, \textbf{2019 - 2024}\vspace{0.25cm}
                
                \noindent Assistant Professor of Mechanical and Biomedical Engineering, The Pennsylvania State University, University Park, PA, \textbf{2013 - 2019}\vspace{0.25cm}
                
    \subsection*{Government}
    
                \noindent \parbox[t]{0.8\linewidth}{\raggedright Mechanical Engineer, The U.S. Army Research Laboratory, Soldier Protection Sciences Branch} \hfill \parbox[t]{0.2\linewidth}{\raggedleft 2009 - 2012} \\
                
                \noindent \parbox[t]{0.8\linewidth}{\raggedright Post-Doc, Oak Ridge Associated Universities at The U.S. Army Research Laboratory, Impact Physics Branch} \hfill \parbox[t]{0.2\linewidth}{\raggedleft 2008 - 2009} \\
                
    \subsection*{Professional}
    
                \noindent \parbox[t]{0.8\linewidth}{\raggedright Founder and Chief Engineer, BrainSim Technologies Inc.} \hfill \parbox[t]{0.2\linewidth}{\raggedleft 2019 - Present} \\
                
                \noindent \parbox[t]{0.8\linewidth}{\raggedright Associate Professor of Biomedical Engineering, The Pennsylvania State University
University Park, PA} \hfill \parbox[t]{0.2\linewidth}{\raggedleft 2019 - 2024} \\
                
                \noindent \parbox[t]{0.8\linewidth}{\raggedright Associate Professor of Mechanical Engineering, The Pennsylvania State University
University Park, PA} \hfill \parbox[t]{0.2\linewidth}{\raggedleft 2019 - 2024} \\
                
                \noindent \parbox[t]{0.8\linewidth}{\raggedright Assistant Professor of Biomedical Engineering, The Pennsylvania State University
University Park, PA} \hfill \parbox[t]{0.2\linewidth}{\raggedleft 2016 - 2019} \\
                
                \noindent \parbox[t]{0.8\linewidth}{\raggedright Assistant Professor of Mechanical Engineering, The Pennsylvania State University
University Park, PA} \hfill \parbox[t]{0.2\linewidth}{\raggedleft 2013 - 2019} \\
                
                \noindent \parbox[t]{0.8\linewidth}{\raggedright Lead Researcher of Computational Biomechanics, The Johns Hopkins University Applied Physics Laboratory, Research and Exploratory Development Department, Biomechanics and Injury Mitigation Systems Group} \hfill \parbox[t]{0.2\linewidth}{\raggedleft 2012 - 2013} \\
                

    \section*{EDUCATION}
    
            \noindent \parbox[t]{0.8\linewidth}{\raggedright Post-Doctoral, U.S. Army Research Laboratory} \hfill \parbox[t]{0.2\linewidth}{\raggedleft 2008 - 2009} \\
            \noindent \parbox[t]{0.8\linewidth}{\raggedright \textbf{Concentration:} Mechanics} \\
            
            \noindent \parbox[t]{0.8\linewidth}{\raggedright Ph D, The Johns Hopkins University} \hfill \parbox[t]{0.2\linewidth}{\raggedleft 2008} \\
            \noindent \parbox[t]{0.8\linewidth}{\raggedright \textbf{Major:} Mechanical Engineering} \\
            
            \noindent \parbox[t]{0.8\linewidth}{\raggedright MS, The Johns Hopkins University} \hfill \parbox[t]{0.2\linewidth}{\raggedleft 2006} \\
            \noindent \parbox[t]{0.8\linewidth}{\raggedright \textbf{Major:} Mechanical Engineering} \\
            
            \noindent \parbox[t]{0.8\linewidth}{\raggedright BS, University of Maryland,  Baltimore County (UMBC)} \hfill \parbox[t]{0.2\linewidth}{\raggedleft 2003} \\
            \noindent \parbox[t]{0.8\linewidth}{\raggedright \textbf{Major:} Mechanical Engineering} \\
            

    \section*{AWARDS AND HONORS}
    
        \noindent \parbox[t]{0.8\linewidth}{\raggedright Outstanding Presentation Award, Penn State Center of Neural Engineering} \hfill \parbox[t]{0.2\linewidth}{\raggedleft 2023} \\
        
        \noindent \parbox[t]{0.8\linewidth}{\raggedright Fellow, American Society of Mechanical Engineering} \hfill \parbox[t]{0.2\linewidth}{\raggedleft 2023} \\
        
        \noindent \parbox[t]{0.8\linewidth}{\raggedright Best Poster Award, Penn State Center of Neural Engineering} \hfill \parbox[t]{0.2\linewidth}{\raggedleft 2022} \\
        
        \noindent \parbox[t]{0.8\linewidth}{\raggedright Faculty Early Career Development (CAREER) Program Award, National Science Foundation} \hfill \parbox[t]{0.2\linewidth}{\raggedleft 2019} \\
        
        \noindent \parbox[t]{0.8\linewidth}{\raggedright First Place Paper and Oral Presentation (presented by my graduate student, I was senior author on paper), 13th Annual Penn State College of Engineering Research Symposium (CERS)} \hfill \parbox[t]{0.2\linewidth}{\raggedleft 2016} \\
        
        \noindent \parbox[t]{0.8\linewidth}{\raggedright People's Choice Poster Award (presented by my student, I was senior author on poster), 2016 (Ernst) Mach Conference} \hfill \parbox[t]{0.2\linewidth}{\raggedleft 2016} \\
        
        \noindent \parbox[t]{0.8\linewidth}{\raggedright Shuman Early Career Professorship, Penn State University Department of Mechanical and Nuclear Engineering} \hfill \parbox[t]{0.2\linewidth}{\raggedleft 2013} \\
        
        \noindent \parbox[t]{0.8\linewidth}{\raggedright Presidential Early Career Awards for Scientists and Engineers (PECASE), The White House; Office of Science and Technology Policy} \hfill \parbox[t]{0.2\linewidth}{\raggedleft 2011} \\
        

    \section*{PUBLICATIONS}
    \textit{Mentored student and postdoc co-authors are underlined.}
    \subsection*{Journal Articles}
    \begin{enumerate}
     \item	\underline{Martin, V.}, Hannah, T., Ellis, S., \&
 \textbf{\textbf{Kraft,} R. H.} (Corresponding Author) (2023). Using the embedded element finite element method to simulate impact of Dyneema plates. Fibers and Polymers. Published. \url{https://doi.org/10.1007/s12221-023-00417-z}
 \item	\underline{Hannah, T.}, \textbf{\textbf{Kraft,} R. H.}, \underline{Martin, V.}, \&
 Ellis, S. Impact of imperfect Kolsky bar experiments across different scales using finite elements. Journal of Verification, Validation and Uncertainty Quantification.
 \item	\underline{Hannah, T.}, Schuster, B., Baker, Z., Ellis, S., \&
 \textbf{\textbf{Kraft,} R. H.} Miniature Kolsky Bar Experiment Techniques Applied to UHMWPE Composite Analysis. Journal of Dynamic Behavior of Materials.
 \item	Zuidema, T. R., Bazarian, J. J., Kercher, K. A., Rettke, D. J., Mannix, R., \textbf{\textbf{Kraft,} R. H.}, Newman, S. D., Ejima, K., Steinfeldt, J. A., \&
 Kawata, K. (2023). Longitudinal association of clinical and biochemical biomarkers with head impact exposure in adolescent football. JAMA Network Open. Published. \url{https://doi.org/10.1001/jamanetworkopen.2023.16601}
 \item	\underline{Menghani, R. R.}, Dasans, A., \&
 \textbf{\textbf{Kraft,} R. H.} (Corresponding Author) (2023). A sensor-enabled cloud-based computing platform for computational brain biomechanics. Computer Methods in Biomechanics and Biomedical Engineering. Published. \url{https://doi.org/10.1016/j.cmpb.2023.107470}
 \item	Ramtani, S., Sanchez, J. F., Boucetta, A., \textbf{\textbf{Kraft,} R. H.}, Vaca-Gonzalez, J. J., \&
 Garzon-Alvarado, D. A. (2023). A coupled mathematical model between bone remodeling and tumors: a study of different scenarios using Komarova's model. Biomechanics and Modeling in Mechanobiology. Published. \url{https://doi.org/10.1007/s10237-023-01689-3}
 \item	Ji, S., Ghajari, M., Mao, H., \textbf{\textbf{Kraft,} R. H.}, Hajiaghamemar, M., Panzer, M. B., Willinger, R., Gilchrist, M. D., Kleiven, S., \&
 Stitzel, J. D. (2022). Use of brain biomechanical models for monitoring impact exposure in contact sports. Annals of Biomedical Engineering. Published. \url{https://doi.org/10.1007/s10439-022-02999-w}
 \item	\underline{Martin, V.}, \textbf{\textbf{Kraft,} R. H.} (Corresponding Author), \underline{Hannah, T.}, \&
 Ellis, S. (2022). An energy-based study of the embedded element method for explicit dynamics. Advanced Modeling and Simulation in Engineering Sciences. Published. \url{https://doi.org/10.1186/s40323-022-00223-x}
 \item	Adewole, D. O., Struzyna, L. A., Harris, J. P., Nemes, A. D., Burrell, J. C., Petrov, D., \textbf{\textbf{Kraft,} R. H.}, Chen, I., Serruya, M. D., Wolf, J. A., \&
 Cullen, K. (2021). Development of optically controlled "living electrodes" with long-projecting axon tracts for a synaptic brain-machine interface. Science Advances 7(4). Published. \url{https://doi.org/10.1126/sciadv.aay5347}
 \item	\underline{Marinov, T.}, \underline{Yuchi, L.}, Adewole, D. O., Cullen, D. Kacy, \&
 \textbf{\textbf{Kraft,} R. H.} (2020). A computational model of bidirectional axonal growth in micro-tissue engineered neuronal networks (micro-TENNs). In Silico Biology 13(3-4), pp. 85-99. Published. \url{https://doi.org/10.3233/ISB-180172}
 \item	\underline{Subramani, A. V.}, \underline{Whitley, P., Garimella, H. T.}, \&
 \textbf{\textbf{Kraft,} R. H.} (2020). Fatigue damage prediction in the annulus of cervical spine intervertebral discs using finite element analysis. Computer Methods in Biomechanics and Biomedical Engineering 23(11), 773-784. Published. \url{https://doi.org/10.1080/10255842.2020.1764545}
 \item	Carrera-Pinzon, A. F., Marquez-Florez, K., \textbf{\textbf{Kraft,} R. H.}, Ramtani, S., \&
 Garzon-Alvarado, D. A. (2019). Computational model of a synovial joint morphogenesis. Biomechanics and Modeling in Mechanobiology, 1--14. Published. \url{https://doi.org/10.1007/s10237-019-01277-4}
 \item	\textbf{\textbf{Kraft,} R. H.} (Author), \underline{Lee, C.}, Richtsmeier, J. T., \&
 \underline{Dolack, M. E.} (2019). Exploring mechanisms of cranial vault development using a coupled turing-biomechanical model. The FASEB Journal 33, 326.2-326.2. Published. \url{https://doi.org/10.1096/fasebj.2019.33.1_supplement.326.2}
 \item	\underline{Lee, C.}, Richtsmeier, J. T., \&
 \textbf{\textbf{Kraft,} R. H.} (2019). A coupled reaction-diffusion-strain model predicts cranial vault formation in development and disease. Biomechanics and Modeling in Mechanobiology. Published. \url{https://doi.org/10.1007/s10237-019-01139-z}
 \item	\underline{Przekwas, A. J., Tan, X. Gary, Chen, Z. J., Miao, Y., Harrand, V., Garimella, H. T.}, \textbf{\textbf{Kraft,} R. H.}, \&
 Gupta, R. K. (2019). Biomechanics of blast TBI with time resolved consecutive primary, secondary and tertiary loads. Military Medicine. Published. \url{https://doi.org/10.1093/milmed/usy344}
 \item	\underline{Garimella, H. T.}, \underline{Menghani, R.}, \underline{Gerber, J. I.}, \underline{Sridhar, S.}, \&
 \textbf{\textbf{Kraft,} R. H.} (2018). Embedded finite elements for modeling axonal injury. Annals of Biomedical Engineering. Published. \url{https://doi.org/10.1007/s10439-018-02166-0}
 \item	\underline{Motiwale, S.}, \underline{Subramani, A. V.}, Zhou, A., \&
 \textbf{\textbf{Kraft,} R. H.} (2018). A non-linear multi-axial fatigue damage model for the cervical intervertebral disc annulus. Advances in Mechanical Engineering 10(6). Published. \url{https://doi.org/10.1177/1687814018779494}
 \item	\underline{Dhobale, A. V.}, \underline{Adewole, O., Chan, A., Marinov, T.}, Serruya, M., \textbf{\textbf{Kraft,} R. H.}, \&
 Cullen, D. Kacy (2018). Assessing functional connectivity across 3D tissue engineered axonal tracts using calcium fluorescence imaging. Journal of Neural Engineering 15(5). Published. \url{https://doi.org/10.1088/1741-2552/aac96d}
 \item	\underline{Ranslow, A.}, \underline{Fang, Z.}, \underline{De Tomas, P.}, Gunnarsson, A., Weerasooriya, T., Satapathy, S., Thompson, K. A., \&
 \textbf{\textbf{Kraft,} R. H.} (2018). The multiaxial failure response of porcine trabecular skull bone estimated using microstructural simulations. American Society of Mechanical Engineers (ASME) Journal of Biomechanical Engineering 140(10). Published. \url{https://doi.org/10.1115/1.4039895}
 \item	\underline{Garimella, H. T.}, \textbf{\textbf{Kraft,} R. H.}, \&
 Przekwas, A. J. (2018). Do blast-induced skull flexures result in axonal deformation? PLOS One 13(3). Published. \url{https://doi.org/10.1371/journal.pone.0190881}
 \item	Serruya, M. D., Harris, J. P., Adewole, D. O., Struzyna, L. A., Burrell, J. C., Nemes, A., Petrov, D., \textbf{\textbf{Kraft,} R. H.}, Chen, H. I., Wolf, J. A., \&
 Cullen, D. K. (2017). Engineered axonal tracts as    "living electrodes" for synaptic-based modulation of neural circuitry. Advanced Functional Materials, 1701183-n/a. Published. \url{https://doi.org/10.1002/adfm.201701183}
 \item	\underline{Lee, C. X.}, Richtsmeier, J. T., \&
 \textbf{\textbf{Kraft,} R. H.} (2017). A computational analysis of bone formation in the cranial vault using a coupled reaction-diffusion-strain model. Journal of Mechanics in Medicine and Biology 17(4). Published. \url{https://doi.org/10.1142/S0219519417500737}
 \item	\underline{Garimella, H. T.}, \&
 \textbf{\textbf{Kraft,} R. H.} (2017). A new computational approach for modeling diffusion tractography in the brain. Journal of Neural Regeneration Research 12(1). Published. \url{https://doi.org/10.4103/1673-5374.198967}
 \item	\underline{Garimella, H. T.}, \&
 \textbf{\textbf{Kraft,} R. H.} (2016). Modeling the mechanics of axonal fiber tracts using the embedded finite element method. International Journal for Numerical Methods in Biomedical Engineering 33(5), 1-21. Published. \url{https://doi.org/10.1002/cnm.2796}
 \item	\underline{Fielding, R. A.}, Przekwas, A. J., Tan, X. G., \&
 \textbf{\textbf{Kraft,} R. H.} (2015). Development of a lower extremity model for high strain rate impact loading. International Journal of Experimental and Computational Biomechanics 3(2), 161-186. Published. \url{https://doi.org/10.1504/IJECB.2015.070427}
 \item	\underline{Lee, C. X.}, Richtsmeier, J. T., \&
 \textbf{\textbf{Kraft,} R. H.} (2015). A computational analysis of bone formation in the cranial vault in the mouse. Frontiers in Bioengineering and Biotechnology 3(24). Published. \url{https://doi.org/10.3389/fbioe.2015.00024}
 \item	Swab, J. J., Tice, J., Wereszczak, A. A., \&
 \textbf{\textbf{Kraft,} R. H.} (2014). Fracture toughness of advanced structural ceramics: Applying ASTM C1421. Journal of the American Ceramic Society, pp. 1-9. Published. \url{https://doi.org/10.1111/jace.13293}
 \item	Clayton, J. D., \textbf{\textbf{Kraft,} R. H.}, \&
 Leavy, R. B. (2012). Mesoscale modeling of nonlinear elasticity and fracture in ceramic polycrystals under dynamic shear and compression. Journal of Solids and Structures 49(18), 6. Published. \url{https://doi.org/10.1016/j.ijsolstr.2012.05.035}
 \item	\textbf{\textbf{Kraft,} R. H.}, Mckee, P. J., Dagro, A. M., \&
 Grafton, S. T. (2012). Combining the finite element method with structural connectome-based analysis for modeling neurotrauma: Connectome neurotrauma mechanics. PLoS Computational Biology 8(8), e1002619. Published. \url{https://doi.org/10.1371%2Fjournal.pcbi.1002619}
 \item	\textbf{\textbf{Kraft,} R. H.}, \&
 Molinari, J. F. (2008). A statistical investigation of the effects of grain boundary properties on transgranular fracture. Acta Materialia 56(17), 10. Published. \url{https://doi.org/10.1016/j.actamat.2008.05.036}
 \item	\textbf{\textbf{Kraft,} R. H.}, Molinari, J. F., Ramesh, K. T., \&
 Warner, D. W. (2008). Computational micromechanics of dynamic compressive loading of a brittle polycrystalline material using a distribution of grain boundary properties. The Journal of Mechanics and Physics of Solids 56, 23. Published. \url{https://doi.org/10.1016/j.jmps.2008.03.009}

    \end{enumerate}
    
    \subsection*{Conference Proceedings}
    \begin{enumerate}
      \item \underline{Hannah, T.}, \textbf{\textbf{Kraft,} R. H.}, \underline{Martin, V.}, \&
 Ellis, S. (2023). Impact of imperfect Kolsky bar experiments across different scales using finite elements. Published. \url{https://doi.org/10.1115/IMECE2022-96816}
  \item \underline{Martin, V.}, \underline{Hannah, T.}, Ellis, S., \&
 \textbf{\textbf{Kraft,} R. H.} (2023). Towards verification and validation of modeling Dyneema using the embedded finite element method. Published. \url{https://doi.org/10.1115/IMECE2022-96784}
  \item \underline{Hannah, T.}, \textbf{\textbf{Kraft,} R. H.}, \underline{Martin, V.}, \&
 Ellis, S. (2021). Implications of statistical spread to experimental analysis in a novel miniature Kolsky bar. Published. \url{https://doi.org/10.1115/IMECE2020-23976}
  \item \underline{Fang, Z.}, \underline{Ranslow, A. N.}, \&
 \textbf{\textbf{Kraft,} R. H.} (2016). Computational micromechanics of trabecular porcine skull bone using the material point method. Volume 3: Biomedical and Biotechnology Engineering Published. \url{https://doi.org/10.1115/IMECE2016-67748}
  \item \underline{Motiwale, S.}, \underline{Subramani, V. V.}, Zhou, X., \&
 \textbf{\textbf{Kraft,} R. H.} (2016). Damage prediction for a cervical spine intervertebral disc. Volume 3: Biomedical and Biotechnology Engineering . Proceedings of the 2016 American Society of Mechanical Engineers Congress and Exposition.  
Phoenix, Arizona, USA, November 11-17, 2016 Published. \url{https://doi.org/10.1115/IMECE2016-67711}
  \item \underline{Chan, A. H. W., Dhobale, A.}, \underline{Adewole, O., Marinov, T.}, \textbf{\textbf{Kraft,} R. H.} (Author), Cullen, D. K., \&
 Serruya, M. (2016). Analysis of spontaneous calcium signals to infer functional connectivity within a novel "living electrode" neural construct. (pp. 1-2). Proceedings of IEEE.  
Philadelphia, PA, USA, December 3, 2016. Published. \url{https://doi.org/10.1109/SPMB.2016.7846870}
  \item \underline{Ranslow, A. N.}, \textbf{\textbf{Kraft,} R. H.}, \underline{Shannon, R.}, De Tomas-\underline{Medina, P.}, Radovitsky, R., Jean, A., Hautefeuille, M. P., Fagan, B., Ziegler, K. A., Weerasooriya, T., Dileonardi, A. M., Gunnarsson, A., \&
 Satapathy, S. (2016). Microstructural analysis of porcine skull bone subjected to impact loading. Volume 3: Biomedical and Biotechnology Engineering Published. \url{https://doi.org/10.1115/IMECE2015-51979}
  \item \underline{Lee, C.}, \&
 \textbf{\textbf{Kraft,} R. H.} (2016). A coupled reaction-diffusion-strain model for bone growth in the cranial vault. Proceedings of the 2016 Summer Biomechanics, Bioengineering and Biotransport Conference (SB3C2016).    
National Harbor, Maryland. June 29 - July 2, 2016
  \item \underline{Ranslow, A. N.}, \&
 \textbf{\textbf{Kraft,} R. H.} (2016). The development of a "fuzzy" yield envelope for trabecular porcine skull bone using numerical simulations. Proceedings of the 2016 Summer Biomechanics, Bioengineering and Biotransport Conference (SB3C2016).    
National Harbor, Maryland. June 29 - July 2, 2016
  \item \underline{Motiwale, S.}, Eppler, W., Hollingsworth, D., Hollingsworth, C., Morgenthau, J., \&
 \textbf{\textbf{Kraft,} R. H.} (2016). Application of neural networks for filtering non-impact transients recorded from biomechanical sensors. Proceedings of the Institute of Electrical and Electronic Engineers (IEEE) International Conference on Biomedical and Health Informatics. (pp. 204 - 207). IEEE.  
Las Vegas, NV. February, 2016. Published. \url{https://doi.org/10.1109/BHI.2016.7455870}
  \item \underline{Reddy, S. N.}, \underline{Fielding, R. A.}, \underline{Robinson, M. J.}, \&
 \textbf{\textbf{Kraft,} R. H.} (2015). A computational study of fracture in the calcaneus under variable impact conditions. Volume 3: Biomedical and Biotechnology Engineering Published. \url{https://doi.org/10.1115/IMECE2015-51984}
  \item \textbf{\textbf{Kraft,} R. H.}, \&
 \underline{Garimella, H. T.} (2015). Embedded finite elements for modeling traumatic axonal injury. Proceedings of the Summer Biomechanics, Bioengineering and Biotransport Conference    (SB3C 2015).    
Snowbird, Utah, June 17-20, 2015
  \item \underline{Fielding, R. A.}, \underline{Tan, X. G., Przekwas, A. J., Kozuch, C. D.}, \&
 \textbf{\textbf{Kraft,} R. H.} (2015). High rate impact to the human calcaneus: A micromechanical analysis. Volume 3: Biomedical and Biotechnology Engineering Published. \url{https://doi.org/10.1115/IMECE2014-38930}
  \item \underline{Garimella, H. T.}, \underline{Yaun, H.}, Johnson, B. D., Slobounov, S., \&
 \textbf{\textbf{Kraft,} R. H.} (2014). Anisotropic constitutive model of human brain with intravoxel heterogeneity of fiber orientation using diffusion spectrum imaging (DSI). Volume 3: Biomedical and Biotechnology Engineering Published. \url{https://doi.org/10.1115/IMECE2014-39107}
  \item \underline{Makwana, A. R.}, \underline{Krishna, A. R.}, \underline{Yuan, H.}, \textbf{\textbf{Kraft,} R. H.}, Zhou, X., Przekwas, A. J., \&
 Whitley, P. (2014). Towards a micromechanical model of intervertebral disc degeneration under cyclic loading. Published. \url{https://doi.org/10.1115/IMECE2014-39174}
  \item \underline{Lee, C.}, Richtsmeier, J. T., \&
 \textbf{\textbf{Kraft,} R. H.} (2014). A multiscale computational model for the growth of the cranial vault in craniosynostosis. Published. \url{https://doi.org/10.1115/IMECE2014-38728}
  \item \underline{Fielding, R. A.}, \textbf{\textbf{Kraft,} R. H.}, Ryan, T. M., \&
 Stecko, T. D. (2014). A micromechanics-based simulation of calcaneus fracture and fragmentation due to impact loading. Proceedings of the 11th World Congress on Computational Mechanics (WCCM XI) 5th. European Conference on Computational Mechanics (ECCM V)    6th. European Conference on Computational Fluid Dynamics (ECFD VI).
  \item Zhang, J., Merkle, A. C., Carneal, C. M., Armiger, R. S., \textbf{\textbf{Kraft,} R. H.}, Ward, E. E., Ott, K. A., Wickwire, A. C., Dooley, C. J., Harrigan, T. P., \&
 Roberts, J. C. (2013). Effects of torso-borne mass and loading severity on early response of the lumbar spine under high-rate vertical loading. International Research Council on Biomechanics of Injury.    
Gothenburg, Sweden, September 11 - 13,    2013
  \item \textbf{\textbf{Kraft,} R. H.}, Dagro, A. M., McKee, P. J., Grafton, S. T., Vettel, J., McDowell, K., Vindiola, M., \&
 Merkle, A. C. (2013). Combining the finite element method with structural network-based analysis for modeling neurotrauma. (pp. 4). 11th International Symposium, Computer Methods in Biomechanics and Biomedical Engineering.
  \item Scheidler, M., Fitzpatrick, J., \&
 \textbf{\textbf{Kraft,} R. H.} (2011). In Tom Proulx (Ed.), Optimal pulse shapes for SHPB tests on soft materials. 1, (pp. 259-268). Society for Experimental Mechanics Series, Dynamic Behavior of Materials.  ISBN/ISSN: 2191-5644 
DOI 10.1007/978-1-4614-0216-9. June, Uncasville, Connecticut. Published. \url{https://doi.org/10.1007/978-1-4614-0216-9_37,}
  \item \textbf{\textbf{Kraft,} R. H.}, Lynch, M. L., \&
 Vogel, E. W. (2011). Computational failure modeling of lower extremities. RTO-MP-HFM-207AC/323(HFM-207)(TP/412) . NATO Human Factors and Medicine Panel. ISBN/ISSN: 978-92-837-0153-8
  \item Clayton, J. D., \&
 \textbf{\textbf{Kraft,} R. H.} (2011). Mesoscale modeling of dynamic failure of ceramic polycrystals. Advances in Ceramic Armor VII: Ceramic Engineering and Science Proceedings. (32), (pp. 237-248). Proceedings of the 35th International Conference on Advanced Ceramics and Composites. ISBN/ISSN: 10.1002/9781118095256.ch21
  \item Vettel, J. M., Bassett, D. S., \textbf{\textbf{Kraft,} R. H.}, \&
 Grafton, S. T. (2010). Physics-based models of brain structure connectivity informed by diffusion weighted imaging. Proceedings of the 27th Army Science Conference.
  \item Gazonas, G. A., McCauley, J. W., \textbf{\textbf{Kraft,} R. H.}, Love, B. M., Clayton, J. D., Casem, D., Dandekar, D., Rice, B., Batyrev, I., Weingarten, N. S., \&
 Schuster, B. E. (2010). Multiscale modeling of armor ceramics: Focus on AlON. 27th Army Science Conference.
  \item Scheidler, M., \&
 \textbf{\textbf{Kraft,} R. H.} (2010). In C. P. Hoppel (Ed.), Inertial effects in compression Hopkinson bar tests on soft materials. U.S. Army Research Laboratory, 1st Annual ARL Ballistic Technology Workshop.
  \item \textbf{\textbf{Kraft,} R. H.}, Batyrev, I., Lee, S., Rollett, A. D., \&
 Rice, B. (2010). In J. J. Swab, S. Mathur and T. Ohji (Eds.), "Multiscale modeling of armor ceramics." Journal of the American Ceramics Society Meeting Proceedings. 31 . Hoboken, NJ: John Wiley \&
 Sons, Inc..  
Ch.13. Published. \url{https://doi.org/10.1002/9780470944004}
  \item Wereszczak, A. A., \&
 \textbf{\textbf{Kraft,} R. H.} (2003). In W. M. Kriven and H. T. Lin (Eds.), Flexural and torsional resonances of ceramic tiles via impulse excitation of vibration. 24(4), (pp. 207-213). 27th Annual Conference on Advanced Ceramics and Composites: B: Ceramic Engineering and Science Proceedings. Published. \url{https://doi.org/10.1002/9780470294826.ch31}
  \item Wereszczak, A. A., \&
 \textbf{\textbf{Kraft,} R. H.} (2002). In H. T. Lin and M. Singh (Eds.), Instrumented Hertzian indentation of armor ceramics. 23(3), (pp. 11). 26th Annual Conference on Composites, Advanced Ceramics, Materials, and Structures: A: Ceramic Engineering and Science Proceedings. Published. \url{https://doi.org/10.1002/9780470294741.ch7}

    \end{enumerate}
    
    \subsection*{Preprints and Technical Reports}
    \begin{enumerate}
      \item \underline{Marinov, T.}, \underline{Yuchi, L.}, Adewole, D. O., Cullen, D. K., \&
 \textbf{\textbf{Kraft,} R. H.} "A computational model of bidirectional axonal growth in micro-tissue engineered neuronal networks (micro-TENNs)." bioRxiv. Cold Spring Harbor Laboratory. Published. \url{https://doi.org/10.1101/369843}
  \item \underline{Gerber, J. I.}, \textbf{\textbf{Kraft,} R. H.}, \&
 \underline{Garimella, H. T.} (2018). "Computation of history-dependent mechanical damage of axonal fiber tracts in the brain: towards tracking sub-concussive and occupational damage to the brain." bioRxiv. Published. \url{https://doi.org/10.1101/346700}
  \item \underline{Garimella, H. T.}, \underline{Menghani, R.}, \underline{Gerber, J. I.}, \underline{Sridhar, S.}, \&
 \textbf{\textbf{Kraft,} R. H.} (2018). "Embedded finite elements for modeling axonal injury." engrXiv. Published. \url{https://doi.org/10.31224/osf.io/2dx5e}
  \item Adewole, D. O., Struzyna, L. A., Harris, J. P., Nemes, A. D., Burrell, J. C., Petrov, D., \textbf{\textbf{Kraft,} R. H.}, Chen, I., Serruya, M. D., Wolf, J. A., \&
 Cullen, K. (2018). "Optically-controlled "living electrodes" with long-projecting axon tracts for a synaptic brain-machine interface." bioRxiv. Published. \url{https://doi.org/10.1101/333526}
  \item Dagro, A. M., McKee, P. J., \textbf{\textbf{Kraft,} R. H.}, Zhang, T. G., \&
 Satapathy, S. S. (2013). A preliminary investigation of traumatically induced axonal injury in a three-dimensional (3-D) finite element model (FEM) of the human head during blast-loading. Army Research Laboratory Technical Report (ARL-TR-6504).
  \item Vettel, J., Dagro, A. M., Gordon, S., Kerick, S., \textbf{\textbf{Kraft,} R. H.}, Luo, S., Rawal, S., Vindiola, M., \&
 McDowell, K. (2012). Brain structure-function couplings (FY11). Army Research Laboratory Technical Report (ARL-TR-5893).
  \item \textbf{\textbf{Kraft,} R. H.}, \&
 Wozniak, S. L. (2011). A review of computational spinal injury biomechanics research and recommendations for future efforts. Army Research Laboratory Technical Report (ARL-TR-5673).
  \item \textbf{\textbf{Kraft,} R. H.}, \&
 Dagro, A. M. (2011). Design and implementation of a numerical technique to inform anisotropic hyperelastic finite element models using diffusion-weighted imaging. Army Research Laboratory Technical Report (ARL-TR-5796).
  \item Clayton, J. D., \&
 \textbf{\textbf{Kraft,} R. H.} (2011). Mesoscale modeling of dynamic failure of ceramic polycrystals. Army Research Laboratory Reprint (ARL-RP-328).
  \item Gozonas, G. A., McCauley, J. W., Batyrev, I. G., Casem, D., Clayton, J. D., Dandekar, D. P., \textbf{\textbf{Kraft,} R. H.}, Love, B. M., Rice, B. M., Schuster, B. E., \&
 Weingarten, N. S. (2011). Multiscale modeling of armor ceramics: Focus on AlON. Army Research Laboratory Reprint (ARL-RP-337).
  \item Vettel, J. M., Bassett, D., \textbf{\textbf{Kraft,} R. H.}, \&
 Grafton, S. (2010). Physics-based models of brain structure connectivity informed by diffusion-weighted imaging. Army Research Laboratory Technical Reprint (ARL-RP-0355). Aberdeen Proving Ground, MD: U.S. Army Research Laboratory.
  \item Wereszczak, A. A., Swab, J. J., \&
 \textbf{\textbf{Kraft,} R. H.} (2005). Effects of machining on the uniaxial and equibiaxial flexure strength of CAP3 AD-995 Al2O3. Army Research Laboratory Technical Report (ARL-TR-3617).
  \item Swab, J. J., Wereszczak, A. A., Tice, J., Caspe, R., \textbf{\textbf{Kraft,} R. H.}, \&
 Adams, J. (2005). Mechanical and thermal properties of advanced ceramics for gun barrel applications. Army Research Laboratory Technical Report (ARL-TR-3417).

    \end{enumerate}
    

    \section*{CONTRACT, FELLOWSHIPS, GRANTS AND SPONSORED RESEARCH}
    
        \noindent Kraft, Reuben H. (Principal Investigator), ``Development of Predictive Disc Degeneration Simulations for Pilots'', Sponsored by Air Force Research Laboratory, \$359,796.00. (September 30, 2022 - September 30, 2025).\vspace{0.25cm}
        
        \noindent Kraft, Reuben H. (Principal Investigator), ``Elucidating High Strain Rate Deformation Mechanisms in Penetration-Resistant Composites'', Sponsored by Triad National Security, LLC (was LANL - Los Alamos National Laboratory), \$214,423.00. (June 16, 2022 - September 30, 2024).\vspace{0.25cm}
        
        \noindent Kraft, Reuben H. (Principal Investigator), ``Examining the link between finite element-based strain predictions and cognitive changes.'', Sponsored by Chuck Noll Foundation [MP], \$77,589.00. (August 1, 2022 - August 1, 2024).\vspace{0.25cm}
        
        \noindent Kraft, Reuben H. (Principal Investigator), ``Unfunded Collaborative Research Agreement - Sports \& Wellbeing Analytics'', Sponsored by Sports \& Wellbeing Analytics, \$1.00. (March 17, 2021 - March 16, 2024).\vspace{0.25cm}
        
        \noindent Kraft, Reuben H. (Principal Investigator), ``CAREER: Multiscale Modeling of Axonal Fiber Bundles in the Brain'', Sponsored by National Science Foundation, \$396,514.00. (February 15, 2019 - January 31, 2024).\vspace{0.25cm}
        
        \noindent Hillman, Michael (Principal Investigator), ``STTR PHASE II Enhancing Thermo-Mechanically Coupled Computational Models for High-Temperature Impact and Fracture'', Sponsored by Karagozian \& Case, Inc., \$179,000.00. (July 1, 2021 - December 2, 2023).\vspace{0.25cm}
        
        \noindent Kraft, Reuben H. (Principal Investigator), ``Occupational mTBI from repeated exposure to low-level blast'', Sponsored by Biokinetics and Associates Ltd, \$111,000.00. (May 1, 2022 - March 31, 2023).\vspace{0.25cm}
        
        \noindent Kraft, Reuben H. (Principal Investigator), ``Development of a Novel Ballistic Armor Concept using FEM'', Sponsored by Triad National Security, LLC (was LANL - Los Alamos National Laboratory), \$56,363.00. (July 24, 2017 - March 31, 2022).\vspace{0.25cm}
        
        \noindent Kraft, Reuben H. (Principal Investigator), ``An Exploration of the Material Point Method (MPM) in CTH Applied to Soft Material Systems Subjected to Dynamic Loading (Continuation)'', Sponsored by Sandia National Laboratories, \$50,000.00. (February 2, 2017 - December 31, 2021).\vspace{0.25cm}
        
        \noindent Kraft, Reuben H. (Principal Investigator), ``Development of Predictive Disc Degeneration Simulations for Pilots'', Sponsored by Air Force Research Laboratory, \$39,315.00. (February 22, 2021 - August 21, 2021).\vspace{0.25cm}
        
        \noindent Kraft, Reuben H. (Principal Investigator), ``Head Kinematics Experimentation and Data Analysis'', Sponsored by SURVICE Engineering Company, LLC, \$5,000.00. (September 1, 2020 - May 31, 2021).\vspace{0.25cm}
        
        \noindent Richtsmeier, Joan T. (Principal Investigator), ``Craniosynostosis Network (formerly award number 0254-3543-4609)'', Sponsored by Icahn School of Medicine at Mount Sinai, \$322,692.00. (February 1, 2016 - January 31, 2021).\vspace{0.25cm}
        
        \noindent Szczesny, Spencer (Principal Investigator), ``Stem Cell Mechanotransduction with Tendon Fatigue'', Sponsored by University of Pittsburgh, \$50,335.00. (July 1, 2019 - June 30, 2020).\vspace{0.25cm}
        
        \noindent Kraft, Reuben H. (Principal Investigator), ``SBIR Phase II:    Global-Local Modeling of Aircraft Occupant Safety Assessment during Ejection (Air Force Phase II SBIR)'', Sponsored by CFD Research Corporation, \$139,494.00. (October 25, 2017 - January 20, 2020).\vspace{0.25cm}
        
        \noindent Kraft, Reuben H. (Principal Investigator), ``Development of Commercial Tools for Brain Modeling'', Sponsored by CFD Research Corporation, \$70,637.00. (November 15, 2017 - September 15, 2019).\vspace{0.25cm}
        
        \noindent Szczesny, Spencer (Principal Investigator), ``Stem Cell Mechanotransduction with Tendon Fatigue'', Sponsored by University of Pittsburgh, \$49,666.00. (July 1, 2018 - June 30, 2019).\vspace{0.25cm}
        
        \noindent Kraft, Reuben H. (Principal Investigator), ``NEUP: Multilayer Composite Fuel Cladding for LWR Performance Enhancement and Severe Accident Tolerance'', Sponsored by Massachusetts Institute of Technology, \$50,000.00. (October 1, 2015 - June 30, 2019).\vspace{0.25cm}
        
        \noindent Kraft, Reuben H. (Principal Investigator), ``Embedded Finite Elements for a Multiscale, Multifunctional Approach for Modeling Axonal Bundles'', Sponsored by SURVICE Engineering Company, LLC, \$114,460.00. (December 13, 2017 - March 13, 2019).\vspace{0.25cm}
        
        \noindent Kraft, Reuben H. (Principal Investigator), ``Biological Living Electrodes Using Tissue Engineered Axonal Tracts to Probe and Modulate the Nervous System (Previously Agreement \#569770)'', Sponsored by Pennsylvania, University of, \$124,527.00. (August 1, 2017 - July 31, 2018).\vspace{0.25cm}
        
        \noindent Kraft, Reuben H. (Principal Investigator), ``STTR Phase I:    Synchronizing Video Imagery with Wearable Sensor Data and Side-by-Side Modeling Software to Develop Healthy Habits in Children'', Sponsored by CoachSafe PlaySafe, LLC, \$132,750.00. (July 1, 2016 - June 30, 2018).\vspace{0.25cm}
        
        \noindent Richtsmeier, Joan T. (Principal Investigator), ``Craniosynostosis Network'', Sponsored by Icahn School of Medicine at Mount Sinai, \$374,032.00. (February 1, 2015 - January 31, 2018).\vspace{0.25cm}
        
        \noindent Kraft, Reuben H. (Principal Investigator), ``Embedded Finite Elements for a Multiscale, Multifunctional Approach for Modeling Axonal Bundles'', Sponsored by IAP Worldwide Services, Inc., \$107,684.00. (October 1, 2016 - September 24, 2017).\vspace{0.25cm}
        
        \noindent Kraft, Reuben H. (Principal Investigator), ``Microstructural Analysis of Porcine Skull Bone Subjected to Impact Loading'', Sponsored by Massachusetts Institute of Technology, \$98,000.00. (July 1, 2014 - September 1, 2017).\vspace{0.25cm}
        
        \noindent Kraft, Reuben H. (Principal Investigator), ``Biological Living Electrodes Using Tissue Engineered Axonal Tracts to Probe and Modulate the Nervous System (Previously Agreement \#568000)'', Sponsored by Pennsylvania, University of, \$122,898.00. (August 1, 2016 - July 31, 2017).\vspace{0.25cm}
        
        \noindent Kraft, Reuben H. (Principal Investigator), ``SBIR: Phase II: A Neck Injury Assessment Tool for Prolonged Wear of Head Supported Mass'', Sponsored by CFD Research Corporation, \$69,086.00. (April 21, 2015 - June 14, 2017).\vspace{0.25cm}
        
        \noindent Kraft, Reuben H. (Principal Investigator), ``SBIR Phase II:    Physics and Physiology Based Human Body Model of Blast Injury and Protection'', Sponsored by CFD Research Corporation, \$100,000.00. (April 1, 2015 - May 31, 2017).\vspace{0.25cm}
        
        \noindent Kraft, Reuben H. (Principal Investigator), ``Global-Local Modeling of Aircraft Occupant Safety Assessment during Ejection (Air Force SBIR Phase I)'', Sponsored by CFD Research Corporation, \$22,000.00. (August 4, 2016 - April 15, 2017).\vspace{0.25cm}
        
        \noindent Kraft, Reuben H. (Principal Investigator), ``An Exploration of the Material Point Method (MPM) in CTH Applied to Soft Material Systems Subjected to Dynamic Loading'', Sponsored by Sandia National Laboratories, \$190,644.00. (January 16, 2015 - December 31, 2016).\vspace{0.25cm}
        
        \noindent Kraft, Reuben H. (Principal Investigator), ``Biological Living Electrodes Using Tissue Engineered Axonal Tracts to Probe and Modulate the Nervous System'', Sponsored by Pennsylvania, University of, \$120,313.00. (September 30, 2015 - July 31, 2016).\vspace{0.25cm}
        
        \noindent Kraft, Reuben H. (Principal Investigator), ``A Neck Injury Assessment Tool for Prolonged Wear of Head Supported Mass'', Sponsored by CFD Research Corporation, \$18,568.00. (January 15, 2014 - August 14, 2014).\vspace{0.25cm}
        
        \noindent Kraft, Reuben H. (Principal Investigator), ``Physics and Physiology Based Human Body Model of Blast Injury and Protection'', Sponsored by CFD Research Corporation, \$36,000.00. (January 7, 2014 - August 6, 2014).\vspace{0.25cm}
        

    \section*{INTELLECTUAL PROPERTY}
    
        \noindent Kraft, R. H. "Brain Simulation Technology."  (application: 2017)\vspace{0.25cm}
        
        \noindent Kraft, R. H. "SmartGear:    Instrumented Wrestling Headgear using Sensors."  (application: 2017)\vspace{0.25cm}
        
        \noindent Kraft, R. H. "Method and Apparatus for Multimodal Mobile Screening to Quantitatively Detect Brain Function Impairment,"  (application: September 2011)\vspace{0.25cm}
        

    \section*{DIRECTED STUDENT LEARNING}
    \subsection*{Ph.D. Dissertation}
    \begin{enumerate}
    
        \item Grube, R., ``High strain rate material response of Dyneema.'', 2023 - Present
        
        \item Reyes, A., ``Modeling of spinal disc degeneration in fighter jet pilots.'', 2022 - Present
        
        \item Menghani, R., ``Sensor enabled, cloud-based modeling of the brain.'', 2017 - Present
        
        \item Martin, V., ``Modeling Armor Composites Undergoing High Strain Rate Deformation.'', 2019 - August 2023
        
        \item Hannah, T., ``Computational and experimental characterization of high strain rate response of Dyneema.'', January 2018 - July 2023
        
        \item Hertel, Z., ``An exploration of the Material Point Method (MPM) in CTH applied to soft material systems subjected to dynamic loading.'', January 2015 - April 2023
        
        \item Subramani, V., ``Modeling of spinal injury under extreme loading Conditions with emphasis on military loading Scenarios - a mathematical fatigue damage model and finite element study.'', November 2015 - August 2020
        
        \item Lee, C., ``A computational analysis of bone formation in the cranial vault using a reaction-diffusion-strain model.'', December 2013 - May 2018
        
        \item Garimella, H., ``An embedded element based human head model to investigate axonal injury.'', September 2013 - June 2017
        
    \end{enumerate}
    
    \subsection*{Master’s Thesis}
    \begin{enumerate}
    
        \item Fournier, N., ``Finite element modeling of gasket interfaces.'', November 2023 - Present
        
        \item Lovett, J., ``Energy based body armor design.'', August 2021 - August 2023
        
        \item Norris, I., ``Computational modeling of spinal degeneration in F35 pilots.'', November 2020 - December 2022
        
        \item Dolack, M., ``Computational morphogenesis of embryonic bone development: past, present, and future.'', September 2017 - May 2019
        
        \item Gerber, J., ``Development of a history-dependent damage model for the brain due to repetitive impacts.'', November 2016 - May 2018
        
        \item Dhobale, A., ``Assessing functional connectivity of micro-tissue engineered neural networks using calcium fluorescence imaging.'', August 2016 - May 2017
        
        \item Yuchi, L., ``A computational model of bidirectional growth for micro-Tissue Engineered Neuronal Networks (micro-TENNs).'', August 2016 - May 2017
        
        \item Fang, Z., ``MPM methods for modeling trabecular bone.'', August 2016 - May 2017
        
        \item Motiwale, S., ``Modeling intervertebral disc degeneration due to cyclic loading.'', January 2015 - May 2016
        
        \item Ranslow, A., ``Microstructural analysis of porcine skull bone subjected to impact loading.'', July 2014 - May 2016
        
        \item Fielding, R., ``Development of a lower extremity model for high strain rate impact loading.'', September 2013 - May 2015
        
    \end{enumerate}
    
    \subsection*{Postdoctoral Mentorship}
    \begin{enumerate}
    
        \item Marinov, T. Computational neuroscience: simulation of micro-tissue engineered neural networks., September 2016 - July 2018
        
    \end{enumerate}
    
    \subsection*{Undergraduate Honors Thesis}
    \begin{enumerate}
    
        \item Brown, B., ``Advanced visualization techniques for brain modeling'', January 2021 - May 2022
        
        \item Mackay, J., ``Brain impact analysis from overpressure sources through machine learning based on explosion simulations and wearable blast gauges'', January 2021 - May 2022
        
        \item Aklilu, O., ``Experimental and computational investigation of correlates of diffusion tensor imaging changes and mechanical strain'', August 2018 - May 2021
        
        \item Katch, L., ``Reverse source localization for identification of overpressure sources based on wearable blast gauges'', August 2019 - April 2020
        
        \item Casey, P., ``Behavior of a modeled hip implant insertion device through finite element analysis'', November 2017 - May 2018
        
        \item De Tomas-Medina, P., ``Modeling the response of neurons subjected to high rate deformation: Comparing simulations to experimental results'', January 2015 - May 2018
        
        \item Borusiewicz, M., ``Quantifying the structure of micro-tissue engineered neural networks'', August 2016 - May 2017
        
        \item Sodha, K., ``Estimating dynamic properties for biological materials: Design, development, and calibration of a desktop miniaturized double-lap shear Kolsky bar'', September 2014 - May 2016
        
        \item Robinson, M., ``The development of an anatomically correct model of calcaneus fracture and fragmentation due to impact loading'', September 2013 - May 2015
        
    \end{enumerate}
    

    \section*{SERVICE}
    \subsection*{College}
    
        \noindent Committee Work, Member, ``Engineering laptop Initiative''. (July 2021 - December 2021)\vspace{0.25cm}
        
        \noindent Committee Work, Member, ``Activity Insight Faculty Users Committee''. (October 2017 - December 2020)\vspace{0.25cm}
        
        \noindent Academic Leadership and Support Work, Member, ``College of Engineering National Science Foundation CAREER Award Winners''. (April 2016)\vspace{0.25cm}
        
        \noindent Competition Judging, Judge, ``College of Engineering Symposium for Undergraduate Research''. (April 2014)\vspace{0.25cm}
        
    \subsection*{Department}
    
        \noindent Committee Work, Committee Member, ``Mechanical Engineering Promotion and Tenure Committee''. (July 2022 - Present)\vspace{0.25cm}
        
        \noindent Committee Work, Chairperson, ``Mechanical Engineering Strategic Plan Tracking Committee''. (August 2023 - May 2024)\vspace{0.25cm}
        
        \noindent Committee Work, Member, ``Promotion and Tenure Committee''. (August 2023 - May 2024)\vspace{0.25cm}
        
        \noindent Committee Work, Member, ``Research Advancement Committee''. (August 2023 - May 2024)\vspace{0.25cm}
        
        \noindent Committee Work, Chairperson, ``Teaching Load Policy Committee''. (September 2022 - May 2023)\vspace{0.25cm}
        
        \noindent Committee Work, Member, ``Department Facilities Committee''. (August 2022 - May 2023)\vspace{0.25cm}
        
        \noindent Committee Work, Member, ``Promotion and Tenure Committee''. (August 2022 - May 2023)\vspace{0.25cm}
        
        \noindent Committee Work, Member, ``Research Advancement Committee''. (August 2022 - May 2023)\vspace{0.25cm}
        
        \noindent Committee Work, Committee Member, ``Mechanical Engineering Strategic Planning Committee''. (January 2022 - December 2022)\vspace{0.25cm}
        
        \noindent Committee Work, Chairperson, ``Teaching Load Policy Committee''. (September 2020 - May 2021)\vspace{0.25cm}
        
        \noindent Committee Work, Member, ``Mechanical Engineering Strategic Planning Committee''. (August 2019 - September 2020)\vspace{0.25cm}
        
        \noindent Committee Work, Chairperson, ``Joint Faculty Search in Mechanical Engineering and the Institute for CyberScience''. (August 2018 - May 2019)\vspace{0.25cm}
        
        \noindent Committee Work, Member, ``Faculty Search Committee for Mechanical Systems in Mechanical Engineering''. (August 2017 - 2018)\vspace{0.25cm}
        
        \noindent Committee Work, Liaison, ``Mechanical Engineering Liaison to Institute for CyberScience''. (2017 - 2018)\vspace{0.25cm}
        
        \noindent Committee Work, Member, ``Faculty Search Committee for Emerging Areas in Mechanical Engineering''. (August 2016 - 2017)\vspace{0.25cm}
        
        \noindent Committee Work, Member, ``Faculty Search Committee for Mechanical Systems in Mechanical Engineering''. (September 2014 - March 2015)\vspace{0.25cm}
        
    \subsection*{University}
    
        \noindent Committee Work, Chairperson, ``Institute for Computational \& Data Sciences Coordinating Committee''. (August 2023 - May 2024)\vspace{0.25cm}
        
        \noindent Committee Work, Committee Member, ``Graduate Council Committee on Academic Standards''. (June 2023 - May 2024)\vspace{0.25cm}
        
        \noindent Committee Work, Committee Member, ``Graduate Council Representative to Engineering Faculty''. (June 2023 - May 2024)\vspace{0.25cm}
        
        \noindent Committee Work, Co-Chairperson, ``Institute for Computational \& Data Sciences Coordinating Committee''. (August 2022 - May 2023)\vspace{0.25cm}
        
        \noindent Committee Work, Member, ``Hiring Committee for Project Coordinator for Institute for CyberScience''. (March 2019 - April 2019)\vspace{0.25cm}
        
        \noindent Participation in Development/Fundraising Activities, Member, ``AI/ML Faculty Engagement Team on behalf of Institute for CyberScience''. (October 2019 - May 2020)\vspace{0.25cm}
        
        \noindent Participation in Development/Fundraising Activities, Member, ``Faculty Participant''. (April 2019)\vspace{0.25cm}
        
    \subsection*{Profession}
    
        \noindent Steering Committee Member (Elected), American Society of Mechanical Engineering International Mechanical Engineering Congress and Exposition (IMECE). (2021 - Present). \vspace{0.25cm}
        
        \noindent Organizing Conferences and Service on Conference Committees, ``Technical Chair (Elected)'', American Society of Mechanical Engineering International Mechanical Engineering Congress and Exposition (IMECE). ((November 2023 - November 2024).)\vspace{0.25cm}
        
        \noindent Organizer for Biological and Biomimetic Soft Materials Symposium, 2024 Mach Conference, Co-Organizer. (April 2024). \vspace{0.25cm}
        
        \noindent Organizing Conferences and Service on Conference Committees, ``Vice Technical Chair (Elected)'', American Society of Mechanical Engineering International Mechanical Engineering Congress and Exposition (IMECE). ((November 2022 - November 2023).)\vspace{0.25cm}
        
        \noindent Chair of Brain and Injury Mechanics Symposium, Brain and Injury Mechanics Symposium, SB3C Conference, Co-Chairperson. (June 2023). \vspace{0.25cm}
        
        \noindent Organizer for Biological and Biomimetic Soft Materials Symposium, 2023 Mach Conference, Co-Organizer. (April 2023). \vspace{0.25cm}
        
        \noindent Primary Organizer and Co-Chairperson, Damage Biomechanics Symposium at the 2022 ASME International Mechanical Engineering Congress and Exposition (IMECE). (November 2021 - November 2022). \vspace{0.25cm}
        
        \noindent Track Co-Chair, Biomedical & Biotechnology Engineering Track at the 2022 ASME International Mechanical Engineering Congress and Exposition (IMECE). (November 2021 - November 2022). \vspace{0.25cm}
        
        \noindent Organizer for Injury Biomechanics Symposium, 2022 Society of Engineering Science (SES) Annual Technical Meeting, Co-Organizer. (February 2022 - October 2022). \vspace{0.25cm}
        
        \noindent Primary Organizer and Co-Chairperson, Damage Biomechanics Symposium at the 2021 ASME International Mechanical Engineering Congress and Exposition (IMECE), Co-Organizer. (November 2020 - November 2021). \vspace{0.25cm}
        
        \noindent Primary Organizer and Co-Chairperson, Damage Biomechanics Symposium at the 2020 ASME International Mechanical Engineering Congress and Exposition (IMECE), Co-Organizer. (November 2019 - November 2020). \vspace{0.25cm}
        
        \noindent Primary Organizer and Co-Chairperson, Damage Biomechanics Symposium at the 2019 ASME International Mechanical Engineering Congress and Exposition (IMECE), Co-Organizer. (November 2018 - November 2019). \vspace{0.25cm}
        
        \noindent Chair of Brain Biomechanics II - Measurement and modeling Symposium, Brain Biomechanics II - Measurement and modeling Symposium, Co-Chairperson. (August 2019). \vspace{0.25cm}
        
        \noindent Chair of Growth Remodeling and Repair II: Musculoskeletal System Symposium, Summer Biomechanics, Bioengineering, and Biotransport (SB3C) Conference, Co-Chairperson. (June 2019).\vspace{0.25cm}
        
        \noindent Primary Organizer and Co-Chairperson, Special symposium on "Computational Modeling of Morphogenesis: Friend or Foe?" at the annual meeting of American Association of Anatomists (AAA), Co-Chairperson. (May 2018 - April 2019). \vspace{0.25cm}
        
        \noindent Primary Organizer and Co-Chairperson, Damage Biomechanics Symposium at 2018 ASME International Mechanical Engineering Congress and Exposition (IMECE), Co-Organizer. (November 2017 - November 2018). \vspace{0.25cm}
        
        \noindent Co-Organizer, "Multiscale Brain Mechanics: From Growth to Injury" Symposium at 18th U.S. National Congress for Theoretical and Applied Mechanics, Co-Organizer. (August 2017 - June 2018). \vspace{0.25cm}
        
        \noindent Primary Organizer and Co-Chairperson, Damage Biomechanics Symposium at the 2017 ASME International Mechanical Engineering Congress and Exposition (IMECE), Co-Organizer. (November 2016 - November 2017). \vspace{0.25cm}
        
        \noindent Primary Organizer and Co-Chairperson, Damage Biomechanics Symposium at the 2016 ASME International Mechanical Engineering Congress and Exposition (IMECE), Co-Organizer. (November 2015 - November 2016). \vspace{0.25cm}
        
        \noindent Co-Chairperson, Brain Injury Symposium at Summer Biomechanics, Bioengineering, and Biotransport (SB3C) Conference, Co-Chairperson. (June 2016).\vspace{0.25cm}
        
        \noindent Co-Organizer and Co-Chairperson, Damage Biomechanics Symposium at the 2015 ASME International Mechanical Engineering Congress and Exposition (IMECE), Co-Organizer. (December 2014 - November 2015). \vspace{0.25cm}
        
        \noindent Co-Organizer and Co-Chairperson, Advances in Computational Biomechanics Symposium at 2015 Pan-American Congress on Computational Mechanics International Conference, Co-Organizer. (June 2014 - June 2015). \vspace{0.25cm}
        
        \noindent Organizing Conferences and Service on Conference Committees, ``Co-Organizer and Co-Chairperson'', 2014 Mid-Atlantic Section (M-AS) of the American Physical Society (APS) ((January 2014 - October 2014).)\vspace{0.25cm}
        
    \subsection*{Society}
    
        \noindent Service to Governmental Agencies, Panelist, ``NDSEG Fellowship Evaluation Panelist'', National Defense Science and Engineering Graduate (NDSEG) Fellowship program. (2015)\vspace{0.25cm}
        

    \section*{EDITORIAL BOARD POSITIONS}
    
        \noindent ASME Journal of Engineering and Science in Medical Diagnostics and Therapy (JESMDT), Associate Editor. (November 2022 - Present).\vspace{0.25cm}
        
        \noindent Frontiers in Bioengineering and Biotechnology, Associate Editor. (November 2014 - Present).\vspace{0.25cm}
        

    \section*{PROFESSIONAL MEMBERSHIPS}
    
        \noindent Member, American Society of Mechanical Engineers. (January 2003 - Present).\vspace{0.25cm}
        
        \noindent Member, United States Association for Computational Mechanics. (February 2014 - 2015).\vspace{0.25cm}
        
        \noindent Member, American Physical Society. (May 2013 - 2014).\vspace{0.25cm}
        
        \noindent Member, American Society for Engineering Education. (May 2013 - 2014).\vspace{0.25cm}
        
        \noindent Member, American Society of Biomechanics. (May 2013 - 2014).\vspace{0.25cm}
        
\end{document}
